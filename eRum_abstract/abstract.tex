\documentclass[a4paper,10pt]{article}
\usepackage[utf8]{inputenc}
\usepackage{natbib}
\bibliographystyle{unsrtnat}
%opening
\title{N-gram analysis of biological sequences in R}
\author{}
\date{}

\begin{document}

\maketitle

N-grams (k-mers) are vectors of $n$ characters derived from input sequences, 
widely used in genomics, transcriptomics and proteomics. Despite the continuous 
interest in the sequence analysis, there are only a few tools tailored for 
comparative n-gram studies. Furthermore, the volume of n-gram data is usually 
very large, making its analysis in \textbf{R} especially challenging. 

The CRAN package \textit{biogram}~\citep{burdukiewicz_biogram:_2015} facilitates 
incorporating n-gram data in the \textbf{R} workflows. Aside from the efficient 
extraction and storage of n-grams, the package offers also a feature selection 
method designed specifically for this type of data. QuiPT (Quick Permutation 
Test) uses several filtering criteria such as information gain (mutual 
information) to choose significant n-grams. To speed up the computation and 
allow precise estimation of small p-values, QuiPT performs an exact test instead 
of a large number of permutations. In addition to this, \textit{biogram} 
contains tools designed for reducing the dimensionality of the amino acid 
alphabet\citep{murphy_simplified_2000}, further scaling down the feature space.

To illustrate the usage of n-gram data in the analysis of biological sequences,  
we present two case studies performed solely in \textbf{R}. The first,   
prediction of amyloids, short proteins associated with the number of clinical   
disorders as Alzheimer's or Creutzfeldt-Jakob’s 
diseases~\citep{fandrich_oligomeric_2012}, employs random 
forests~\citep{wright_ranger:_2015} trained on n-grams. The second, detection   
of signal peptides orchestrating an extracellular transport of proteins, 
utilizes more complicated probabilistic framework (Hidden semi-Markov 
model,) but still uses n-gram data for training. 

\bibliography{amyloids}

\end{document}
